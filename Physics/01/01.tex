\documentclass[portuguese,11pt,a4paper,oneside,openany,notitlepage]{article}
\usepackage[T1]{fontenc}
\usepackage{babel}
\usepackage{graphicx}
\title{Áreas da Física}
\author{mr.spoopy}
\begin{document}
	\maketitle
	
	\section{As principais áreas da \textbf{física}}
	
	A física é dividida em várias áreas de estudo, cada uma responsável por investigar diferentes aspectos da natureza. Vou dividir em Física Clássica \textit{x} Física Moderna \textit{e} Físicas Interdisciplinares, \newline A \textbf{física Clássica} é toda a física desenvolvida antes do século XX, ela descreve muito bem o mundo cotidiano. \newline A \textbf{física Moderna} é a física que surgiu do século XX em diante, para explicar o que a física clássica nao conseguia, coisas MUITO grandes e MUITO pequenas \newline As \textbf{físicas Interdisciplinares} São áreas da física que se misturam com outras ciências para resolver problemas complexos ou aplicar a física em outros contextos.
	
	% TODO: \usepackage{graphicx} required
	\begin{center}
		\includegraphics[width=1\linewidth]{physics}
	\end{center}
	
	
	\subsection{Física Clássica}
	
	\subsubsection{Mecânica}
	A Mecânica é o estudo do movimento dos corpos, a força que os causam e os princípios de equilíbrio. Divide-se em:
	
	\begin{center}
		\includegraphics[width=1\linewidth]{cinematica}
	\end{center}
	
	\paragraph{Cinemática}
	Estuda o movimento dos corpos sem considerar suas causas. Analisa grandezas como: \textit{posição}, \textit{velocidade} e \textit{aceleração}, utilizando equações matemáticas para descrever trajetórias.\newline É fundamental para engenharia de transportes, como no cálculo da trajetória de foguetes. \newline (ex: movimento parabólico de projéteis ou a queda livre de objetos)
	
	\begin{center}
		\includegraphics[width=1\linewidth]{dinamica}
	\end{center}
	
	\paragraph{Dinâmica}
	Investiga as forças que causam ou modificam o movimento, baseando-se nas leis de Newton. Explica fenômenos como a inércia, a relação entre força e aceleração e a lei da ação e reação. Aplica-se desde o funcionamento de motores até a análise de múltiplas colisões. \newline (ex: Peritos físicos analisam o "nível" de amasso do carro para calcular a que velocidade o carro que bateu estava)
	
	\begin{center}
		\includegraphics[width=1\linewidth]{estatica}
	\end{center}
	
	\paragraph{Estática}
	Dedica-se ao equilíbrio de corpos rígidos, onde forças e torques se cancelam. Essencial na construção civil, garantindo que não haja movimento indesejado. \newline (ex: criação de pontes e edifícios)
	
	\begin{center}
		\includegraphics[width=1\linewidth]{fluido}
	\end{center}
	
	\paragraph{Mecânica dos Fluidos}
	Examina o comportamento de líquidos e gases em repouso(hidrostática) e em movimento(hidrodinâmica). Conceitos como pressão, viscosidade e empuxo são cruciais para projetos de submarinos, sistema de irrigação e aerodinâmica de aviões \newline (ex: efeito Bernoulli)
	
	\begin{center}
		\includegraphics[width=1\linewidth]{termo}
	\end{center}
	
	\subsubsection{Termodinâmica}
	Estuda a relação entre calor, trabalho, temperatura e energia. Suas quatro leis explicam, como acontece a conversão de calor em energia mecânica (1\textordfeminine{} lei) e o porque alguns processos naturais só acontecem de um jeito e não dá para voltar atrás (2\textordfeminine lei, entropia). Também explica mudanças de fase (fusão, vaporização). \newline (ex: criação/estudo de motores térmicos)
	
	\begin{center}
		\includegraphics[width=1\linewidth]{optica}
	\end{center}
	
	\subsubsection{Óptica}
	Estuda a luz e suas interações com a matéria. Divide-se em:
	\paragraph{Geométrica}
	Analisa a luz como raios que seguem trajetórias retilíneas, explicando fenômenos como reflexão em espelhos e refração em lentes. \newline (ex: instrumentos como microscópios e telescópios)
	\paragraph{Óptica Ondulatória}
	Mostra que a luz se comporta como uma onda, criando efeitos como anéis coloridos, padrões ao passar por fendas e luz filtrada. Isso é usado em hologramas e óculos com filtro de luz. \newline (ex: quando você olha para uma bolha de sabão e vê aquelas cores brilhantes e variadas, interferência)
	
	\begin{center}
		\includegraphics[width=1\linewidth]{eletromagnetismo}
	\end{center}
	
	\subsubsection{Eletromagnetismo}
	Estuda fenômenos relacionados à eletricidade e ao magnetismo, mostrando que oss dois estão ligados, Dividi-se em:
	\paragraph{Eletrostática}
	Estuda cargas elétricas em repouso, descrevendo campos elétricos, potencial eletrostático e a força entre as cargas (Lei de Coulomb). Aplica-se a isolantes, capacitores e etc. \newline (ex: estudo de descargas atmosféricas como raios)
	\paragraph{Circuitos}
	Foca na corrente elétrica, explorando resistência, leis de Ohm e Kirchoff. Essencial para o funcionamento de dispositivos eletrônicos e redes de energia. \newline (ex: Dentro de qualquer computador) 
	\paragraph{Magnetostática}
	A magnetostática é a parte do eletromagnetismo que estuda os campos magnéticos gerados por correntes elétricas constantes (que não mudam com o tempo). \newline (ex: no alto falante á um imã)
	\paragraph{Eletrodinâmica}
	A eletrodinâmica estuda como os campos elétricos e magnéticos mudam com o tempo e como eles se influenciam mutuamente.Essa troca constante é o que forma as ondas eletromagnéticas. \newline (ex: ondas de rádio, micro-ondas, wifi, 4G, etc...)
	
	\begin{center}
		\includegraphics[width=1\linewidth]{ondulatoria}
	\end{center}
	
	\subsubsection{Ondulatória}
	Estuda as propriedades das ondas, sejam mecânicas (som, ondas em cordas) ou eletromagnéticas (luz, rádio). Abrange conceitos como frequência, comprimento de onda, ressonância e superposição. \newline (ex: análise de terremotos, ondas sísmicas)
	
	\begin{center}
		\includegraphics[width=1\linewidth]{ondulatoria(1)}
	\end{center}
	
	\subsubsection{Acústica}
	Especializada em ondas sonoras, investiga propriedades como intensidade (decibéis), frequência (tons graves/agudos) e timbre.Aplica-se ao projeto de instrumentos musicais, isolamento acústico em construções e tecnologias médicas (ultrassom). \newline (ex: Nos radares metereológicos é utilizado o efeito doppler)
	
	\subsection{Física Moderna}
	
	\begin{center}
		\includegraphics[width=1\linewidth]{relatividade}
	\end{center}
	
	\subsubsection{Relatividade Restrita}
	Desenvolvida por Einstein em 1905, descreve objetos em movimento próximo à velocidade da luz. Introduz conceitos como dilatação temporal e contração espacial. \newline (ex: aceleradores de partículas, aceleram as partículas para entender como o tempo é dilatado)
	
	\subsubsection{Relatividade Geral}
	Explica a gravidade como curvatura do espaço-tempo, prevendo buracos negros, ondas gravitacionais (detectadas em 2015) e a expansão do universo. \newline (ex: A Engenharia Aerospacial utiliza satélites de GPS que estão sujeitos a efeitos gravitacionais que alteram a passagem do tempo, e com a relatividade Geral faz com que funcione com precisão)
	
	\begin{center}
		\includegraphics[width=1\linewidth]{quantica}
	\end{center}
	
	\subsubsection{Física Quântica}
	Mudou muito o jeito de entender o mundo das coisas muito pequenas, mostrando que partículas podem estar em vários lugares ou estados ao mesmo tempo (superposição) e que elas podem ficar conectadas de um jeito especial, mesmo estando longe uma da outra (entrelaçamento quântico). \newline (ex: Dentro do LED, acontece um fenômeno chamado emissão eletroluminescente, que só pode ser explicado pela física quântica)
	
	\begin{center}
		\includegraphics[width=1\linewidth]{nuclear}
	\end{center}
	
	\subsubsection{Física Nuclear}
	Investiga o núcleo atômico, suas forças (nuclear forte) e processos como fissão (quebra de núcleos pesados, usada em reatores) e fusão (união de núcleos leves, como no Sol).  \newline (ex: Radioterapia é um tratamento médico que usa radiação ionizante para destruir células cancerígenas)
	
	\subsubsection{Física de Partículas}
	Ela estuda as menores partículas que formam tudo no universo, como quarks, elétrons e neutrinos, e as forças que fazem essas partículas interagirem — tipo eletricidade e forças dentro do átomo.
	Um modelo chamado Modelo Padrão explica como essas partículas e forças funcionam, mas ainda há mistérios, como a matéria escura, que ninguém sabe direito o que é.
	Para entender melhor, cientistas usam máquinas gigantes como o LHC (Grande Colisor de Hádrons) no CERN, que bate partículas muito rápido para criar condições parecidas com o início do universo (Big Bang) e testar essas ideias. \newline (ex: Pesquisas em física de partículas ajudam a entender melhor a estrutura da matéria em nível fundamental. Isso impulsiona o desenvolvimento de materiais supercondutores e nanomateriais usados em eletrônica, energia e transportes.)
	
	\subsection{Fisícas Interdisciplinares}
	
	\begin{center}
		\includegraphics[width=1\linewidth]{cosmologia}
	\end{center}
	
	\subsubsection{astrofísica}
	Estuda o espaço e os corpos celestes, como estrelas, galáxias e buracos negros. Usa telescópios e dados de sondas espaciais (como o Hubble) para entender como esses objetos se formam, funcionam e evoluem.
	
	\subsubsection{Cosmologia}
	É a parte da física que tenta entender o universo inteiro — como ele começou (Big Bang), como está estruturado e o que pode acontecer no futuro. Também estuda mistérios como a matéria escura e a energia escura, que formam a maior parte do universo, mas ainda são pouco compreendidas.
	
	\subsubsection{Biofísica}
	Mistura física com biologia para entender como funcionam os seres vivos. Por exemplo: como moléculas se movem dentro das células, como os músculos se contraem ou como os neurônios transmitem sinais elétricos.
	
	\subsubsection{Física Médica}
	Aplica a física na medicina. Desenvolve equipamentos como raios-X, tomografia e radioterapia. Também estuda a quantidade segura de radiação para evitar danos ao corpo.
	
	\subsubsection{Física Computacional}
	Usa computadores potentes para simular coisas muito complexas, como o clima da Terra, colisões entre galáxias ou testes com novos materiais. Isso ajuda na pesquisa de energia, saúde e até pandemias.
	
	\subsubsection{Física estatística}
	Estuda sistemas com muitas partículas (como gases ou metais) usando probabilidade. Explica fenômenos como mudanças de estado (ex: gelo virando vapor) e como os átomos se organizam para formar ímãs.
	
	\subsubsection{Geofísica}
	Usa a física para entender o planeta Terra: terremotos, vulcões, campo magnético, e o que existe no interior do planeta. É útil em estudos ambientais e na busca por petróleo ou minerais.
	
	\subsubsection{Física do estado sólido}
	Estuda os materiais usados na tecnologia, como semicondutores (chips de computador), supercondutores (transporte de eletricidade sem perdas) e LEDs (luz eficiente). É a base da eletrônica moderna.
	
	\subsubsection{Física da matéria condensada}
	Vai além dos estados comuns da matéria (sólido, líquido, gás). Estuda materiais especiais, como grafeno (ultrafino e resistente), superfluidos e cristais líquidos (usados em telas de TV e celular).
	
	\subsubsection{Física do solo}
	Analisa como água, ar, calor e forças atuam no solo. Ajuda a entender o funcionamento do solo em plantações, no meio ambiente e em construções. Mistura física com geologia e agricultura.
\end{document}