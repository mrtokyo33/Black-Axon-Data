\documentclass[portuguese,11pt,a4paper,oneside, openany]{article}
\usepackage[T1]{fontenc}
\usepackage{babel}
\usepackage{graphicx}
\usepackage{textcomp}

\title{Introdução à Física}
\author{mr.spoopy}
\date{}

\begin{document}
	
	\maketitle

	\begin{center}
		\includegraphics[width=1\linewidth]{historicafísicos}
	\end{center}
	

	Olá, meu nome é spoopy, e sou um grande entusiasta em física e exatas. Nesse primeiro capitulo da seção de \textbf{Física} eu irei fazer uma breve introdução à física.
	
	Nessa Itrodução, eu irei abordar alguns assuntos como:
	\begin{itemize}
		\item O que é \textbf{física}?
		\item Como se divide a \textbf{física}
		\item Origens filosóficas
		\item A relação entre \textbf{Física} e \textbf{Matemática}
	\end{itemize}
	
	
	
	\newpage
	\section{O que é física?}
	A física é a ciência que estuda os fenômenos naturais de todo o universo, tudo aquilo que pode ser observado, medido e descrito em termos de leis e modelos matemáticos.
	
	Seu objetivo principal é entender como e por que as coisas acontecem, desde os movimentos de planetas, até o comportamento de partículas subatômicas, passando pela luz, o som, a eletricidade, o tempo e até o próprio espaço.
	
	\begin{center}
		\includegraphics[width=1\linewidth]{formulas}
	\end{center}
	
	\textbf{Em termos simples:}
	\begin{quote}
		Física é o estudo das leis da natureza que explicam como o universo funciona.
	\end{quote}
	
	\textbf{Curiosidade:}
	\begin{quote}
		O nome "Física" vem do grego \textit{physis}, que significa "natureza".
	\end{quote}
	
	
	
	\newpage
	\section{Como se divide a \textbf{física}}
	
	A física pode ser dividida em três grandes áreas:
	
	\begin{center}
		\includegraphics[width=1\linewidth]{classics}
	\end{center}
	
	\begin{itemize}
		\item \textbf{Física Clássica:} estuda os fenômenos em escalas do cotidiano, como o movimento, a força, a energia, o som e a luz, usando as leis de Newton, a termodinâmica e o eletromagnetismo.
		
		\begin{center}
			\includegraphics[width=1\linewidth]{modern}
		\end{center}
		
		\item \textbf{Física Moderna:} aborda fenômenos em escalas muito pequenas (átomos e partículas) ou muito rápidas e energéticas. Inclui a relatividade, a mecânica quântica, a física nuclear e de partículas.
		
		\begin{center}
			\includegraphics[width=1\linewidth]{astro}
		\end{center}
		
		\item \textbf{Físicas Interdisciplinares:} misturam a física com outras áreas, como:
		\begin{itemize}
			\item \textit{Astrofísica e Cosmologia} (universo e estrelas),
			\item \textit{Biofísica} (fenômenos em seres vivos),
			\item \textit{Física Médica} (imagens e tratamentos),
			\item \textit{Geofísica} (estrutura da Terra),
			\item \textit{Física Computacional} (simulações com computadores),
			\item \textit{Física do Solo}, entre outras.
		\end{itemize}
	\end{itemize}
	
	Essas divisões ajudam a organizar o conhecimento, mas muitas vezes se sobrepõem, já que a física está presente em tudo ao nosso redor.
	
	\section{Origens filosóficas}
	
	A física nasceu da filosofia. Na Grécia Antiga, pensadores como \textbf{Tales de Mileto}, \textbf{Pitágoras}, \textbf{Platão} e \textbf{Aristóteles} tentavam entender a natureza usando a razão, sem depender de mitos.
	
	Esses filósofos buscavam responder perguntas como:
	
	\begin{itemize}
		\item Do que tudo é feito?
		\item Por que as coisas se movem?
		\item Existe uma ordem na natureza?
	\end{itemize}
	
	Com o tempo, essa busca por entender o mundo evoluiu para uma ciência baseada em observação, experimentação e matemática.
	
	\textbf{Exemplo:} Aristóteles acreditava que objetos caíam mais rápido se fossem mais pesados. Séculos depois, \textbf{Galileu Galilei} provou com experimentos que isso não é verdade — um marco na transição da filosofia para a física moderna.
	
	\begin{center}
		\includegraphics[width=1\linewidth]{aristoteles-galileu}
	\end{center}
	
	A física, portanto, tem raízes filosóficas profundas, mas se tornou uma ciência independente quando passou a se basear em testes e medidas.
	
	
	
	\section{A relação entre \textbf{Física} e \textbf{Matemática}}
	
	A \textbf{física} e a \textbf{matemática} estão profundamente conectadas. A física busca entender como o universo funciona, enquanto a matemática fornece a linguagem para descrever e prever esses fenômenos com precisão.
	
	\subsection*{Por que a física usa matemática?}
	
	A natureza segue padrões e leis. A matemática permite:
	
	\begin{itemize}
		\item \textbf{Expressar leis físicas de forma exata} — como $F = m a$ (Segunda Lei de Newton).
		\item \textbf{Fazer previsões} — como calcular a trajetória de um foguete.
		\item \textbf{Analisar fenômenos complexos} — como ondas, circuitos elétricos ou o comportamento de átomos.
	\end{itemize}
	
	\subsection*{Exemplo prático}
	
	Imagine estudar o movimento de um carro. A física observa o movimento; a matemática permite calcular a velocidade, a aceleração e o tempo para parar. Sem matemática, seria impossível prever com segurança esses comportamentos.
	
	\begin{center}
		\includegraphics[width=1\linewidth]{fisica-e-matematica}
	\end{center}
	
	\subsection*{Matemática como ferramenta da física}
	
	\begin{itemize}
		\item \textbf{Álgebra e trigonometria} — fundamentais para resolver problemas de mecânica e óptica.
		\item \textbf{Cálculo diferencial e integral} — usados para estudar movimento, variações de energia e campos.
		\item \textbf{Equações diferenciais} — modelam sistemas como o pêndulo, circuitos elétricos ou a propagação de ondas.
		\item \textbf{Geometria e vetores} — importantes para representar forças, deslocamentos e campos físicos.
		\item \textbf{Estatística e probabilidade} — usadas em física quântica, termodinâmica e física de partículas.
	\end{itemize}
	
	\subsection*{Resumo}
	
	A matemática não apenas ajuda a descrever fenômenos físicos — ela é essencial para a própria existência da física como ciência. Sem matemática, a física seria apenas uma observação sem previsão.
	
	\subsection*{Curiosidade}
	
	O físico \textbf{Galileu Galilei} dizia:
	
	\begin{quote}
		\textit{“A matemática é o alfabeto com o qual Deus escreveu o universo.”}
	\end{quote}
	
	
\end{document}