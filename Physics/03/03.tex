\documentclass[portuguese,11pt,a4paper,oneside,onecolumn,notitlepage]{article}
\usepackage[T1]{fontenc}
\usepackage[utf8]{inputenc}
\usepackage{babel}
\usepackage{graphicx}
\usepackage{amsmath}
\usepackage{amssymb}
\usepackage{float}
\usepackage{array}
\usepackage{booktabs}
\usepackage{geometry}
\usepackage{fancyhdr}
\usepackage{titlesec}
\usepackage{multicol}
\usepackage{hyperref}
\geometry{a4paper, margin=2.5cm}
\titleformat{\section}{\normalfont\Large\bfseries}{\thesection}{1em}{}
\titleformat{\subsection}{\normalfont\large\bfseries}{\thesubsection}{1em}{}

\title{\textbf{Unidades de Medida e o Sistema Internacional (SI)}}
\author{mr.spoopy}
\date{}

\pagestyle{fancy}
\fancyhf{}
\rhead{Física Básica}
\lhead{Unidades de Medida}
\rfoot{\thepage}

\begin{document}
	\maketitle
	
	\section{Por que Padronizar as Unidades de Medida?}
	
	Desde os tempos antigos, diferentes civilizações utilizaram unidades próprias para medir comprimento, massa e tempo. Por exemplo, os egípcios usavam o "côvado", os romanos o "pé", e os franceses utilizavam múltiplas unidades locais. Isso causava sérios problemas em trocas comerciais, engenharia, navegação e ciência.
	
	Com o avanço da ciência e da globalização, tornou-se imprescindível ter um \textbf{sistema unificado e universal}. As vantagens da padronização incluem:
	
	\begin{itemize}
		\item \textbf{Precisão científica}: permite que experimentos possam ser reproduzidos em qualquer lugar do mundo.
		\item \textbf{Facilidade de comunicação}: evita mal-entendidos entre cientistas, engenheiros, profissionais da saúde e técnicos.
		\item \textbf{Eficiência comercial}: facilita o comércio internacional, pois todos utilizam as mesmas referências.
		\item \textbf{Segurança}: erros de unidade podem causar acidentes (como o famoso caso da sonda Mars Climate Orbiter, perdida por confusão entre milhas e quilômetros).
	\end{itemize}
	
	\section{O que é o Sistema Internacional (SI)?}
	
	O \textbf{Sistema Internacional de Unidades (SI)} foi criado em 1960, durante a 11ª Conferência Geral de Pesos e Medidas (CGPM). Ele é o sucessor do sistema métrico decimal e tem como base sete unidades fundamentais.
	
	\subsection*{Características do SI}
	\begin{itemize}
		\item É \textbf{coerente}: as unidades derivadas resultam logicamente da combinação das fundamentais.
		\item É \textbf{decimal}: as unidades e seus múltiplos/submúltiplos usam potências de 10.
		\item É \textbf{universal}: adotado oficialmente por quase todos os países.
		\item É \textbf{dinâmico}: atualiza-se conforme os avanços da ciência (por exemplo, a redefinição do quilograma em 2019).
	\end{itemize}
	
	\subsection*{Unidades Fundamentais do SI}
	
	\begin{table}[H]
		\centering
		\begin{tabular}{>{\bfseries}l l l l}
			\toprule
			Grandeza Física & Unidade SI & Símbolo & Definição Atual \\
			\midrule
			Comprimento & metro & m & Distância percorrida pela luz em 1/299\,792\,458 s \\
			Massa & quilograma & kg & Definido pela constante de Planck \\
			Tempo & segundo & s & 9\,192\,631\,770 períodos da radiação do césio-133 \\
			Corrente elétrica & ampère & A & Definido pela carga elementar $e$ \\
			Temperatura termodinâmica & kelvin & K & Definido pela constante de Boltzmann \\
			Quantidade de substância & mol & mol & 6{,}02214076 × $10^{23}$ entidades elementares \\
			Intensidade luminosa & candela & cd & Intensidade de radiação em uma direção específica \\
			\bottomrule
		\end{tabular}
		\caption{Grandezas Fundamentais do Sistema Internacional}
	\end{table}
	
	\section{Prefixos do SI}
	
	Os prefixos são utilizados para representar múltiplos e submúltiplos das unidades, tornando mais prático o uso de números muito grandes ou muito pequenos.
	
	\subsection*{Principais Prefixos (Mais Usados)}
	
	\begin{table}[H]
		\centering
		\begin{tabular}{>{\bfseries}l l l l}
			\toprule
			Nome & Símbolo & Fator & Exemplo \\
			\midrule
			Quilo & k & $10^3$ & 1 km = 1\,000 m \\
			Hecto & h & $10^2$ & 1 hL = 100 L \\
			Deca & da & $10^1$ & 1 dam = 10 m \\
			(de unidade) & – & $10^0$ & 1 m = 1 m \\
			Deci & d & $10^{-1}$ & 1 dm = 0{,}1 m \\
			Centi & c & $10^{-2}$ & 1 cm = 0{,}01 m \\
			Mili & m & $10^{-3}$ & 1 mm = 0{,}001 m \\
			Micro & µ & $10^{-6}$ & 1 µm = $10^{-6}$ m \\
			Nano & n & $10^{-9}$ & 1 nm = $10^{-9}$ m \\
			Pico & p & $10^{-12}$ & 1 pF = $10^{-12}$ F \\
			\bottomrule
		\end{tabular}
		\caption{Prefixos Decimais do SI}
	\end{table}
	
	\subsection*{Prefixos Menos Usados (para ciência avançada)}
	
	\begin{itemize}
		\item \textbf{Giga (G)} = $10^9$ — usado em informática (1 GB = $10^9$ bytes)
		\item \textbf{Tera (T)} = $10^{12}$ — 1 terawatt = $10^{12}$ watts
		\item \textbf{Femto (f)} = $10^{-15}$ — usado em escalas atômicas
		\item \textbf{Atto (a)} = $10^{-18}$ — tempo de decaimento de partículas subatômicas
	\end{itemize}
	
	\section{Unidades Derivadas no SI}
	
	\subsection*{Exemplos Comuns de Unidades Derivadas}
	
	\begin{table}[H]
		\centering
		\begin{tabular}{>{\bfseries}l l l}
			\toprule
			Grandeza Derivada & Unidade & Expressão \\
			\midrule
			Velocidade & metro por segundo & m/s \\
			Aceleração & metro por segundo ao quadrado & m/s² \\
			Força & newton & N = kg·m/s² \\
			Trabalho/Energia & joule & J = N·m = kg·m²/s² \\
			Potência & watt & W = J/s = kg·m²/s³ \\
			Pressão & pascal & Pa = N/m² = kg/m·s² \\
			Carga elétrica & coulomb & C = A·s \\
			Tensão elétrica & volt & V = W/A = kg·m²/s³·A \\
			\bottomrule
		\end{tabular}
		\caption{Unidades Derivadas do SI}
	\end{table}
	
	\subsection*{Observação}
	As unidades derivadas também podem possuir \textbf{nomes próprios}, como joule, newton, watt, em homenagem a cientistas que contribuíram para essas áreas.
	
	\section{Aplicações Práticas das Unidades SI}
	
	\subsection*{Física}
	\begin{itemize}
		\item Medição da velocidade da luz: $3{,}0 \times 10^8$ m/s
		\item Força de atrito entre duas superfícies: medida em newtons (N)
	\end{itemize}
	
	\subsection*{Química}
	\begin{itemize}
		\item Quantidade de moléculas em 1 mol: $6{,}022 \times 10^{23}$
		\item Volume molar de um gás ideal: 22{,}4 L/mol (em CNTP)
	\end{itemize}
	
	\subsection*{Engenharia}
	\begin{itemize}
		\item Potência de motores: expressa em kilowatts (kW)
		\item Pressão em sistemas hidráulicos: medida em pascal (Pa) ou MPa
	\end{itemize}
	
	\subsection*{Informática}
	\begin{itemize}
		\item Armazenamento digital: megabytes (MB), gigabytes (GB), terabytes (TB)
		\item Velocidade de transmissão de dados: megabits por segundo (Mbps)
	\end{itemize}
	
	\section{Conversão de Unidades}
	
	\subsection*{Regras Gerais}
	\begin{itemize}
		\item Multiplique ou divida por potências de 10 ao converter entre prefixos.
		\item Mantenha a consistência de unidades ao fazer cálculos físicos.
	\end{itemize}
	
	\subsection*{Exemplo de Conversão}
	
	Converter 5 km para metros:
	
	\[
	5\, \text{km} = 5 \times 10^3\, \text{m} = 5\,000\, \text{m}
	\]
	
	Converter 250 mL para litros:
	
	\[
	250\, \text{mL} = 250 \times 10^{-3}\, \text{L} = 0{,}25\, \text{L}
	\]
	
	\section{Curiosidades e Evolução Histórica}
	
	\begin{itemize}
		\item O metro já foi definido como a décima milionésima parte da distância entre o equador e o polo norte.
		\item O quilograma, antes de 2019, era definido por um cilindro de platina-irídio guardado na França.
		\item Atualmente, todas as unidades SI são definidas com base em \textbf{constantes fundamentais da natureza}, como a velocidade da luz ($c$) e a constante de Planck ($h$).
	\end{itemize}
	
	\section{Conclusão}
	
	As unidades de medida e o Sistema Internacional são ferramentas indispensáveis para a ciência moderna. Elas proporcionam precisão, clareza e confiabilidade em todas as áreas do conhecimento humano. A compreensão e aplicação correta dessas unidades são habilidades essenciais para estudantes, pesquisadores e profissionais de qualquer área técnica ou científica.
	
	\vspace{0.5cm}
	\textit{“Medir é saber.” — Lord Kelvin}
	
\end{document}
