\documentclass[portuguese,11pt,a4paper,oneside,onecolumn,notitlepage]{article}
\usepackage[T1]{fontenc}
\usepackage[utf8]{inputenc}
\usepackage{babel}
\usepackage{graphicx}
\usepackage{amsmath}
\usepackage{amssymb}
\usepackage{float}
\usepackage{array}
\usepackage{booktabs}
\usepackage{geometry}
\usepackage{fancyhdr}
\usepackage{titlesec}

\geometry{a4paper, margin=2.5cm}
\titleformat{\section}{\normalfont\Large\bfseries}{\thesection}{1em}{}
\titleformat{\subsection}{\normalfont\large\bfseries}{\thesubsection}{1em}{}

\title{\textbf{Grandezas Físicas e Suas Classificações}}
\author{mr.spoopy}
\date{}

\pagestyle{fancy}
\fancyhf{}
\rhead{Física Básica}
\lhead{Grandezas Físicas}
\rfoot{\thepage}

\begin{document}
	\maketitle
	
	\section{Introdução}
	Na Física, utilizamos as \textbf{grandezas físicas} para descrever os fenômenos naturais e representar medições do mundo real. Toda medição envolve uma \textit{quantidade numérica} e uma \textit{unidade de medida}. Por exemplo, quando dizemos que um corpo tem massa de 50 kg, estamos lidando com uma grandeza física.
	
	\section{O que são Grandezas Físicas?}
	Grandezas físicas são propriedades mensuráveis de um sistema físico. Cada grandeza é expressa por um número seguido de uma unidade. Por exemplo:
	\begin{itemize}
		\item Tempo: 10 segundos
		\item Comprimento: 5 metros
		\item Temperatura: 25 graus Celsius
	\end{itemize}
	
	Essas medições são fundamentais para a ciência, engenharia e nosso dia a dia.
	
	\section{Sistema Internacional de Unidades (SI)}
	O \textbf{Sistema Internacional de Unidades (SI)} é o sistema padrão adotado em quase todos os países para a padronização das unidades de medida. Ele define sete grandezas fundamentais a partir das quais todas as outras são derivadas.
	
	\subsection*{Grandezas Fundamentais e suas Unidades no SI}
	
	\begin{table}[H]
		\centering
		\begin{tabular}{>{\bfseries}l l l}
			\toprule
			Grandeza Física & Unidade SI & Símbolo da Unidade \\
			\midrule
			Comprimento     & metro              & m \\
			Massa           & quilograma         & kg \\
			Tempo           & segundo            & s \\
			Corrente elétrica & ampère           & A \\
			Temperatura termodinâmica & kelvin & K \\
			Quantidade de substância & mol        & mol \\
			Intensidade luminosa & candela         & cd \\
			\bottomrule
		\end{tabular}
		\caption{Grandezas Fundamentais no Sistema Internacional}
	\end{table}
	
	\section{Classificação das Grandezas Físicas}
	
	\subsection{1. Quanto à Dependência de Outras Grandezas}
	
	\subsubsection*{a) Grandezas Fundamentais}
	São aquelas definidas de forma independente. Não dependem de outras grandezas. Ex: tempo, massa, comprimento.
	
	\subsubsection*{b) Grandezas Derivadas}
	São aquelas que dependem de duas ou mais grandezas fundamentais. São obtidas por meio de operações matemáticas. Exemplos:
	\begin{itemize}
		\item Velocidade = comprimento / tempo (m/s)
		\item Força = massa × aceleração (kg·m/s²)
		\item Área = comprimento × comprimento (m²)
	\end{itemize}
	
	\subsection{2. Quanto à Natureza Física}
	
	\subsubsection*{a) Grandezas Escalares}
	Grandezas que são totalmente determinadas apenas por um valor numérico e unidade. Não têm direção nem sentido. Exemplos:
	\begin{itemize}
		\item Temperatura: 37 ºC
		\item Massa: 10 kg
		\item Tempo: 2 h
	\end{itemize}
	
	\subsubsection*{b) Grandezas Vetoriais}
	Grandezas que possuem valor numérico, unidade, direção e sentido. São representadas por vetores. Exemplos:
	\begin{itemize}
		\item Força: 20 N para a direita
		\item Velocidade: 60 km/h para o norte
		\item Aceleração: 9{,}8 m/s² para baixo
	\end{itemize}
	
	\subsection{3. Outras Classificações}
	
	\subsubsection*{Grandezas Extensivas e Intensivas (aplicado em Química e Termodinâmica)}
	\begin{itemize}
		\item \textbf{Extensivas}: dependem da quantidade de matéria. Ex: massa, volume.
		\item \textbf{Intensivas}: não dependem da quantidade de matéria. Ex: temperatura, densidade.
	\end{itemize}
	
	\section{Exemplos Práticos}
	\begin{itemize}
		\item Um carro anda 100 km em 2 horas. Sua velocidade média é uma \textbf{grandeza derivada vetorial}.
		\item Uma pedra tem massa de 2 kg. Massa é uma \textbf{grandeza fundamental escalar}.
		\item A temperatura de um forno é 200 ºC. Temperatura é uma \textbf{grandeza escalar e fundamental}.
	\end{itemize}
	
	\section{Unidades Derivadas}
	As unidades derivadas são combinações das unidades fundamentais. Veja alguns exemplos:
	
	\begin{table}[H]
		\centering
		\begin{tabular}{>{\bfseries}l l l}
			\toprule
			Grandeza Derivada & Unidade Derivada & Expressão SI \\
			\midrule
			Velocidade         & metro por segundo    & m/s \\
			Aceleração         & metro por segundo ao quadrado & m/s² \\
			Força              & newton                & kg·m/s² \\
			Trabalho/Energia   & joule                 & N·m ou kg·m²/s² \\
			Potência           & watt                  & J/s ou kg·m²/s³ \\
			\bottomrule
		\end{tabular}
		\caption{Exemplos de Unidades Derivadas}
	\end{table}
	
	\section{Conclusão}
	O estudo das grandezas físicas é a base para compreender os fenômenos da natureza. Saber classificá-las corretamente permite interpretar melhor os resultados de experimentos, realizar cálculos com precisão e comunicar informações científicas de forma clara e padronizada. Dominar essas noções é fundamental para qualquer estudo em Física e em outras ciências.
	
\end{document}
